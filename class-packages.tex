\documentclass{beamer}
\usepackage{fontspec,polyglossia,xunicode}
\setmainfont{Linux Libertine O}
\setmainlanguage{french}
\usepackage{minted}

\author{Maïeul Rouquette}
\date{Rencontres \LaTeX\ et SHS}
\title{Créer ses propres classes et packages}
\institute{Université de Lausanne --- IRSB}
\usetheme{Darmstadt}
\begin{document}


\begin{frame}
	\titlepage
	\vfill
	{\tiny Licence Creative Commons France 3.0 - Paternité - Partage à l'identique}
\end{frame}


\begin{frame}
	\frametitle{Définitions}
	\begin{description}
		\item[Classe]{Ensemble de commandes et réglages \TeX/\LaTeX\ correspondant à un type éditorial}
		\item[Package]{Ensemble de commandes \TeX correspondant à un besoin fonctionnel}
	\end{description}
\end{frame}

\begin{frame}
	\frametitle{Principes}
	\begin{itemize}
		\item<1->Ensemble de commandes groupée dans un fichier \verb+.cls+ ou \verb+.sty+
		\item<2->Particularité : les commandes peuvent contenir un \verb+@+ dans le nom.
	\end{itemize}
\end{frame}
\begin{frame}
	\frametitle{Avant toutes choses}
	\begin{itemize}
		\item<1->Définir les besoins
			\begin{itemize}
				\item<2->Quels seront les commandes internes (avec \verb+@+) et quels seront les commandes externes (utilisable par les utilisateurs).
				\item<3->Dans l'idéal, rédiger la documentation utilisateur d'abord !
			\end{itemize}
		\item<4->Avoir de la documentation sur \LaTeX\ autre qu'une simple introduction. Voir sur \TeX.
		\item<5->Utiliser un système de gestion des versions (SVN, GIT)
	\end{itemize}
\end{frame}
\end{document}