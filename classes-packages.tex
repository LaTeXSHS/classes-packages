\documentclass{beamer}
\usepackage{fontspec,polyglossia,xunicode,hyperref}
\setmainfont{Linux Libertine O}
\setmainlanguage{french}
\usepackage{minted}
\newcounter{code}
\renewcommand{\thecode}{\arabic{code}.~}
\newcommand{\code}[3]{\only<#1>{\scriptsize\stepcounter{code}\begin{block}{\thecode#2}\inputminted{latex}{code/#3.tex}\end{block}}}

\author{Maïeul Rouquette}
\date{Rencontres \LaTeX\ et SHS}
\title{Créer ses propres classes et packages}
\institute{Université de Lausanne --- IRSB}

\usepackage{biblatex-philologue/biblatex-philologue}
\bibliography{classes-packages.bib}
\renewbibmacro*{cite}{%
  \usebibmacro{cite:citepages}%
    {\usebibmacro{cite:full}}}

\usetheme{Darmstadt}
\begin{document}


\begin{frame}
	\titlepage
	\vfill
	{\tiny Licence Creative Commons France 3.0 - Paternité - Partage à l'identique}
\end{frame}


\begin{frame}
	\frametitle{Définitions}
	\begin{description}
		\item[Classe]Ensemble de commandes et réglages \TeX/\LaTeX\ correspondant à un type éditorial
		\item[Package]Ensemble de commandes \TeX\ correspondant à un besoin fonctionnel
	\end{description}
\end{frame}

\begin{frame}
	\frametitle{Principes}
	\begin{itemize}
		\item<1->Ensemble de commandes groupée dans un fichier .cls ou .sty.
		\item<2->Particularité : les commandes peuvent contenir un @ dans le nom.
	\end{itemize}
\end{frame}

\begin{frame}
	\frametitle{Avant toutes choses}
	\begin{itemize}
		\item<1->Définir les besoins
			\begin{itemize}
				\item<2->Quels seront les commandes internes (avec @) et quels seront les commandes externes (utilisable par les utilisateurs).
				\item<3->Dans l'idéal, rédiger la documentation utilisateur d'abord !
			\end{itemize}
		\item<4->Avoir de la documentation sur \LaTeX\ autre qu'une simple introduction. Voir sur \TeX.
		\item<5->Utiliser un système de gestion des versions (SVN, GIT)
	\end{itemize}
\end{frame}

\begin{frame}
	\frametitle{Commandes de bases}
	\begin{overprint}
		\code{1}{Indiquer le nom, la date et la version (classe)}{infos-classe}
		\code{1}{Indiquer le nom, la date et la version (package)}{infos-package}
		\code{2}{Charger une classe déjà existante}{charger-classe}
		\code{2}{Charger un package déjà existant}{charger-package}			
		\code{3}{Déclarer des options}{declarer-options}
	\end{overprint}
\end{frame}
\begin{frame}
	\frametitle{Quelques packages utiles}
	\begin{itemize}
		\item Pour la gestion de la mise en page
			\begin{itemize}
				\item\alert<1>{geometry}
				\item\alert<2>{fancyhdr}
				\item\alert<3>{TikZ}
			\end{itemize}
		\item Pour créer facilement des commandes
			\begin{itemize}
				\item\alert<4>{xargs}
				\item\alert<5>{xkeyval}
				\item\alert<6>{etoolbox}
			\end{itemize}
	\end{itemize}
\end{frame}

\begin{frame}
	\frametitle{Quelques codes utiles}
	\only<1>{\setcounter{code}{0}}
	\begin{overprint}
		\code{1}{Retenir une valeur fournie par l'utilisateur}{retenir-valeur}
		\code{2}{Tester l'existence d'une commande}{tester-existence}
		\code{3}{Boucler sur une liste de valeurs}{boucler-valeurs}
	\end{overprint}
\end{frame}
\begin{frame}
	\frametitle{Pour aller plus loin}
	\begin{itemize}
		\item<1-> On peut à partir d'un même fichier produire la documentation, le code et la documentation du code, voir : \cite{Pakin2004}
		\item<2-> Un peu de biblio (hors documentation des packages)
			\begin{itemize}
				\item \cite{latex_companion}
				\item \cite{frama}
			\end{itemize}
		\item<3-> Et puis publier sur le CTAN : \url{http://www.ctan.org/upload}
	\end{itemize}
\end{frame}
\end{document}